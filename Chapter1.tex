\chapter{}

\section{Introduction}
	With the ever increasing availability of high resolution three-dimensional scanning and geographic information science software, new methods of lithic morphometric research are rapidly become both possible and practical. If it is truly the goal of the archaeologist to thoroughly and accurately record as much information as possible about their findings, then adopting these methods will soon become the new standard. Utilizing these technological resources, researchers at Oregon State University have developed GLiMR: GIS-based Lithic Morphometric Research. This software utilizes a number of ArcGIS tools to generate a large set of morphometric data which can also later be used for high resolution characterizations, analysis, and comparisons of artifact form.
	This research will use three dimensional digital scanning and GLiMR to examine an assemblage of lithic artifacts from the Pilcher Creek site. This assemblage includes complete projectile points of varying styles, as well as projectile point fragments, bifaces, and other miscellaneous lithic objects. The projectile points fall mostly into two different classifications, including stemmed and lanceolate forms. The benefit of performing this analysis is that a large number of morphometric measurements can be taken, which would be prohibitively time consuming to obtain by hand. These measurements can then be used to validate the original classification, possibly find new classifications, and allow for rapid comparison with assemblages from other sites.

\section{Significance}
	This is where I will talk about the significance of this work, and what contributions it will make to archaeological research.

\section{Research Goals and Questions}
	The primary goal of this research is to use GLiMR to create a data rich, quantitative description of select lithic artifacts from Pilcher creek, which can be used in future cross site comparisons. In order to accomplish this, there are a number of secondary goals which must be met, which include the following. Determine which morphological features best separate the artifacts into distinguishable groups. Determine how many different classification groups the lithic artifacts fall into based on their morphological features. Lastly, determine if it is possible to also classify artifact fragments along with the complete artifacts.
