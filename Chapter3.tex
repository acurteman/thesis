\chapter{}

\section{Artifact Selection}
	

\section{Creation of the Three-dimensional Models}
	The first step in digital lithic morphometric analysis is to create a high resolution three-dimensional models of the desired artifacts. For this research, this was accomplished using a David SLS-2 Structured Light 3D Scanner and software. There are a number of benefits to using a structured light scanner. Compared to other 3D scanning techniques, such as laser scanners or photogrammetry, structured light scanners are relatively fast, create high resolution 3D models, and are not cost prohibitive.
	However, structured light scanners are not capable to detecting dark materials, or transparent materials, and many of the artifacts in this research fall into one or more of those categories. In order to solve this problem, a small amount of Tenactin powder spray applied was applied to the artifacts before scanning. This small white powder coating solves both the problem of artifacts being too dark, and with some obsidian and chert artifacts the problem of being partially transparent.
	The final product from the David Scanning software is a complete three dimensional model in the form of an obj file.

\section{GLiMR}


\section{Statistical Analyses}
